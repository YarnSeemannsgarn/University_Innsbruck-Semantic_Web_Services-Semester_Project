%%%%%%%%%%%%%%%%%%%%%%%%%%%%%%%%%%%%%%%%%
% Short Sectioned Assignment
% LaTeX Template
% Version 1.0 (5/5/12)
%
% This template has been downloaded from:
% http://www.LaTeXTemplates.com
%
% Original author:
% Frits Wenneker (http://www.howtotex.com)
%
% License:
% CC BY-NC-SA 3.0 (http://creativecommons.org/licenses/by-nc-sa/3.0/)
%
%%%%%%%%%%%%%%%%%%%%%%%%%%%%%%%%%%%%%%%%%

%----------------------------------------------------------------------------------------
%	PACKAGES AND OTHER DOCUMENT CONFIGURATIONS
%----------------------------------------------------------------------------------------

\documentclass[paper=a4, fontsize=12pt]{scrartcl} % A4 paper and 11pt font size

\usepackage[T1]{fontenc} % Use 8-bit encoding that has 256 glyphs

\usepackage[english]{babel} % English language/hyphenation
\usepackage{amsmath,amsfonts,amsthm} % Math packages

\usepackage{lipsum} % Used for inserting dummy 'Lorem ipsum' text into the template

\usepackage{sectsty} % Allows customizing section commands
\allsectionsfont{\centering \normalfont\scshape} % Make all sections centered, the default font and small caps

\usepackage{fancyhdr} % Custom headers and footers
\usepackage{hyperref}

\pagestyle{fancyplain} % Makes all pages in the document conform to the custom headers and footers
\fancyhead{} % No page header - if you want one, create it in the same way as the footers below
\fancyfoot[L]{} % Empty left footer
\fancyfoot[C]{} % Empty center footer
\fancyfoot[R]{\thepage} % Page numbering for right footer
\renewcommand{\headrulewidth}{0pt} % Remove header underlines
\renewcommand{\footrulewidth}{0pt} % Remove footer underlines
\setlength{\headheight}{13.6pt} % Customize the height of the header

\numberwithin{equation}{section} % Number equations within sections (i.e. 1.1, 1.2, 2.1, 2.2 instead of 1, 2, 3, 4)
\numberwithin{figure}{section} % Number figures within sections (i.e. 1.1, 1.2, 2.1, 2.2 instead of 1, 2, 3, 4)
\numberwithin{table}{section} % Number tables within sections (i.e. 1.1, 1.2, 2.1, 2.2 instead of 1, 2, 3, 4)

\setlength\parindent{0pt} % Removes all indentation from paragraphs - comment this line for an assignment with lots of text

%----------------------------------------------------------------------------------------
%	TITLE SECTION
%----------------------------------------------------------------------------------------

\newcommand{\horrule}[1]{\rule{\linewidth}{#1}} % Create horizontal rule command with 1 argument of height

\title{	
\normalfont \normalsize 
\textsc{University of Innsbruck} \\ [25pt] % Your university, school and/or department name(s)
\horrule{0.5pt} \\[0.4cm] % Thin top horizontal rule
\huge Semantic Web Services - Semester Project (Kick-off Task) \\ % The assignment title
\horrule{2pt} \\[0.5cm] % Thick bottom horizontal rule
}

\author{Jan Schlenker \& Richard Dvorsky} % Your name

\date{\normalsize\today} % Today's date or a custom date

\begin{document}

\maketitle % Print the title

%----------------------------------------------------------------------------------------
%	SECTION 1
%----------------------------------------------------------------------------------------

\section{Project description}
\label{sec:project_description}

The project goal is to build a JSON-LD generator as a form based tool similiar to the \href{http://www.ldodds.com/foaf/foaf-a-matic}{FOAF-o-matic} tool. To reach this the latest JSON-LD syntax described in \href{https://www.w3.org/TR/json-ld/}{https://\allowbreak{}www.\allowbreak{}w3.\allowbreak{}org/\allowbreak{}TR/\allowbreak{}json-ld/} will be used. The generator should generate the \href{http://schema.org}{http://schema.org} types and their properties which are listed in chapter \ref{sec:types_and_properties}. All of them are in the context of a touristic service.\\
\\
There are already similiar generators like \href{https://www.jamesdflynn.com/development/json-ld-markup-generator/}{https://\allowbreak{}www.\allowbreak{}jamesdflynn.\allowbreak{}com/\allowbreak{}development/\allowbreak{}json-ld-markup-generator/}, which will help to get an idea for the final implementation. More implementation details are mentioned in chapter \ref{sec:implementation_details}.

%----------------------------------------------------------------------------------------
%	SECTION 2
%----------------------------------------------------------------------------------------

\section{Selected schema.org types and properties}
\label{sec:types_and_properties}

The following types with their properties can be generated with the JSON-LD generator:

\begin{description}
\item[\href{https://schema.org/Event}{Event}] The Event type is necessary in the context of tourism, because there are a lot of events like festivals which can be part of a touristic service. It combines locations and times.
\item[\href{https://schema.org/Place}{Place}] Places define where a touristic service takes place. They are also required for the event type. The two most important more specific places for this project are \href{https://schema.org/LocalBusiness}{LocalBusiness} and \href{https://schema.org/TouristAttraction}{TouristAttraction}.
\item[\href{https://schema.org/Review}{Review}] Reviews are necessary to evaluate and to choose a touristic service.
\item[\href{https://schema.org/Offer}{Offer}] Offers are entities ``to transfer some rights to an item or to provide a service'', so they fit perfectly for the project.
\item[\href{https://schema.org/Person}{Person}] Persons utilise touristic services.
\item[\href{https://schema.org/Organization}{Organization}] Organizations offer touristic services. One important more specific type is also the \href{https://schema.org/LocalBusiness}{LocalBusiness} type.
\end{description}

%----------------------------------------------------------------------------------------
%	SECTION 3
%----------------------------------------------------------------------------------------

\section{Implementation details}
\label{sec:implementation_details}

The current project state will be managed via Git on GitHub (\href{https://github.com/YarnSeemannsgarn/University_Innsbuck-Semantic_Web_Services-Semester_Project}{https://\allowbreak{}github.\allowbreak{}com/\allowbreak{}YarnSeemannsgarn/\allowbreak{}University\_Innsbuck-\allowbreak{}Semantic\_Web\_Services-\allowbreak{}Semester\_Project}).\\
\\
The form based tool will be a website, so at least the languages HTML and JavaScript will be used to create the generator. Additional languages, frameworks etc. may be used depending on the state of the project.

%------------------------------------------------

%----------------------------------------------------------------------------------------

\end{document}
